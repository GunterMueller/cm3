\section{Project Architecture}
\subsection{Team Structure}
\begin{description}
\item[Project Lead]
     Coordinates all aspects of the project.
     
\item[Project Architect]
     Approves the initial definition of and the subsequent
     modifications of the public interface, the directory
     structure, and the sufficiency of the supporting
     analysis.
     
\item[Domain Expert]
     Identifies topic area, researches it, develops
     candidate interfaces, writes the analysis, codes the
     test module and the implementation module.
     Periodically, the domain expert reviews his/her work
     with other team members and makes appropriate
     modifications.
     
\item[Publicist]
     Maintains web page, coordinates newsgroup postings,
     coordinates public announcements and discussions of the
     project.
\end{description}

\subsection{Communications}
The team communicates via email and via postings to the
project's web page under
\begin{verbatim}
    http://www.eskimo.com/~hgeorge/m3na
\end{verbatim}

Text must be readable in raw ascii form (e.g., as a simple
insertion into an email message.) The formal documentation is in \TeX.

Generally, a domain expert's analysis needs to be understood
and approved by at least one other team member before the
implementation code is reviewed.  The code, and its test
modules, must be reviewed by at least one other team member
prior to release.

Once the project is established, releases will be made via
M. Dagenais's ftp site.

\subsection{Work Breakdown Structure (WBS)}
\begin{verbatim}
     1.                Manage project
          1.1.         Develop general documentation
          1.2.         Maintain web site
          1.3.         Coordinate analysis and design reviews
          1.4.         Coordinate code reviews
     2.                Develop na.i3 and nap.i3 skeletons
     3.                Develop release <rr>
          3.1.         Identify domain experts and select topics
          3.2.         Develop topic <tt>
               3.2.1.  Research the field, and discuss opportunities
               3.2.2.  Develop test module
               3.2.3.  Perform original analysis, prototype as needed
               3.2.4.  Discuss and approve anaysis.
               3.2.5.  Code up alternatives
               3.2.6.  Test for accuracy
               3.2.7.  Test for timing
               3.2.8.  Review selected alternative with team
               3.2.9.  Stabilize interface, and retest
          3.3.         Publish for external review
          3.4.         Package and release
          3.5.         Perform regression tests
          3.6.         Update user's guide
          3.7.         Update developer's guide
          3.8.         Perform beta tests (if testers are available)
          3.9.         Publicize
\end{verbatim}

\section{Library Architecture}
\subsection{Packaging}
The library is a collection of interface and implementation
modules (x{\it name}.i3/m3).  [Originally, names were in the DOS 8.3
format, but as Spencer Allain noted, we can use long names
if we can tar it all up -- which is now the approach.]

The library is documented with an analysis paper for each
implementation module.  Trivial implementations are not
discussed, but significant algorithms and design tradeoffs
are explained in considerable detail.  The idea is to
demonstrate independent development of the {\em code}, when
working from published algorithms which may have associated
copyrighted code in their own right.

The test suite is composed of a general driver and a
collection of test modules, one for each implementation
module.  Test modules are named the same as the tested
module, with "t" prepended instead of "x".

The initial environment has been NT386, though the code is
not OS dependent.  The file hierarchy is
\begin{verbatim}
     m3na_x.html            main web page
     doc
          backgnd.html      background page
          people.html       participants
          arch.doc          project architecture
          concepts.doc      general numerical analysis
          bib.doc           bibliography
          <name>.doc        documentation for module nn
          
     m3na
          m.bat              m3build | more
          next.bat           cd ..\test
          nt386
               (binaries)
          src
               m3makefile
               template.m3   template for a .m3 module
               na.i3         public interface
               nap.i3        private interface
               x<name>.m3    implementation modules
     
     test
          m.bat              m3build | more
          go.bat             nt386\Test
          next.bat           cd ..\m3na
          nt386
               (binaries)
          src
               m3makefile
               template.m3    template for a .m3 module
               test.i3        interface for all modules
               test.m3        main module and driver
               t<name>.m3     test modules
\end{verbatim}

\subsection{Module Templates}
For {\tt m3na/src} there are {\tt template.i3/m3} which can be instantiated
by copying to and then replacing XYZ with the new module name.

For test/src there is {\tt template.m3} which can be similarly copied and
edited to make new test modules.  Also, that template has an ABC proc
template.

\subsection{Group Structure}
A group is (roughly) a set of elements and a set of
operations which can be applied to them.  We have used this
paradigm to collect functions into objects.  While purists
may quibble about groups vs semigroups, etc., the point is
to distinguish the elements of a group from the group
itself.  This a) allows non-object elements, b) avoids
Smalltalkian absurdities like sending "+" to "3".

In the case of language primitives, the elements can be
literals or variables, and the operations are infix
operators.  In the case of constructed types, we have to
make up our own treatment of literals and operators.  For
those, here are the desired operations:
\begin{itemize}
   \item new (make one)
   \item copy (given one, make another one)
   \item lex (make one by reading a text string)
   \item fmt (use one to make a text string)
   \item eq,ne,gt,ge,lt,le (relational operators) [possibly not
     relevant]
   \item add,sub,mul,div (matrixh operators)
\end{itemize}

A group usually also has several distinguished elements,
such as:
\begin{itemize}
  \item Zero (such that x+Zero = x)
  \item One (such that x*One = x)
\end{itemize}

\subsection{Naming}
Names perform two basic functions:
\begin{enumerate}
  \item Assure lexical uniqueness within scope.
  \item Provide mnemonic assistance to reader
\end{enumerate}

The names will be within Module or Object contexts, so their
names will generally be unique.  Where needed, we provide
added naming conventions.  Uniqueness is a function of both
base type and construction.  Thus, an array of 32-bit reals
is distinguished from an array of 64-bit reals.  And an
array of reals is distinguished from a matrix of reals.  So
for the primitives (which will not be in objects), we need a
naming convention:
\begin{tt} \begin{verbatim}
     prefix means    example
     ------ -------  ---------
     b      bits     bReverseArray = a function to bit-reverse an array
     i      INTEGER  iVector = a vector of INTEGERS
     
     c      COMPLEX  cVector = a vector of COMPLEX (64-bit reals)
     
\end{verbatim} \end{tt}

The result is that a client can do:
\begin{tt} \begin{verbatim}
IMPORT Complex AS C;

VAR c1,c2,c3:C.Complex;           (*make some variables*)
BEGIN
  c1:=C.COMPLEX{re:=1.5,im:=3.3}; (*make a new constant*)
  c2:=C.Zero;                     (*use a predefined constant*)
  c3:=C.add(c1,c2);               (*use an object method*)
END;
\end{verbatim} \end{tt}

\subsection{Arrays}
An array is a collection of items, all of the same type,
which can be accessed by an index into the array.  Array
types need to be distinguished by cell type, which we do
with prefixes such as c and i.  They also need to be
distinguished if their meaning is specific.  For example, a
Vector is not really a general purpose array -- Instead, it
has known operations such as dot product.  Similarly, a
polynomial might be stored in an array, but it has specific
operations such as evaluate at x, or take the derivative.

We could distinguish these special uses of arrays by
BRANDing them.  [I tried that for a while.]  But it is
convenient to allow structural type matrixching, so that one
"fmt" command works for simple arrays as well as Vectors and
rows of Matrices.  Therefore all these types are being
implemented as arrays.  Modula-3 provides several ways to
implement the concept of an array.  We are generally using
dynamic arrays.  There is (or may be) some overhead for
dynamic linking, and some for allocation, but the payoff in
readability is high.

In a few cases (e.g., interpolation tables), we can expect
the array to be a statisticic or constant.  So the appropriate
formal is an open array of REAL64.  If a dynamic array
happens to be used as the actual parm, it can be
dereferenced prior to the call:

\begin{tt} \begin{verbatim}
     y:=Interpolation.linear(xarr^,yarr^,x);
\end{verbatim} \end{tt}

Modula-3's dynamic arrays only range 0..n-1.  But many matrixh
notations call for 1..n or perhaps some offset into the
array.  To provide syntactic support for the semantics, we
use this convention:
\begin{tt} \begin{verbatim}
     PROCEDURE myproc(a1,a2:Array;.......)=
     (*a1 and a2 should be same size*)
     VAR
       n:=NUMBER(a1^);  n1:=0;  nn:=n-1;
     BEGIN
       IF NUMBER(a2^)#n THEN
         RAISE Error(Err.bad_size);
       END;
     
       FOR i:=n1 TO nn DO
         (*do something to the a1's and a2's*)
       END;
     END;
\end{verbatim} \end{tt}

If an offset and/or limited length is needed:
\begin{tt} \begin{verbatim}
     PROCEDURE myproc(a1,a2:Array;
                      offset:CARDINAL;
                      length:CARDINAL:=0;
                      ...
     VAR
       n:=NUMBER(a1^);  n1:=0;  nn:=n-1;
     BEGIN
       IF length#0 THEN
         IF offset+length>n THEN
           RAISE Error(Err.bad_size);
         ELSE
           n:=length; n1:=offset; nn:=n1+n-1;
         END;
       END;
     
       IF NUMBER(a2^)#n THEN
         RAISE Error(Err.bad_size);
       END;
     
       FOR i:=n1 TO nn DO
         (*do something to all the a1's and a2's*)
       END;
     END;
\end{verbatim} \end{tt}

If you seriously need to eliminate the overhead of the
dynamic references, then you can convert to statisticic arrays:
\begin{itemize}
   \item Copy the offending code to a new module
   \item Make a new hardcoded type:
       {\tt TYPE Vector4 = ARRAY [1..4] OF REAL64;}
   \item Change the proc signature to use that type instead
       of the dynamic type
   \item Change the initializations to:
       {\tt VAR n:=NUMBER(a1); n1:=FIRST(a1); nn:=LAST(nn);}
       (notice we no longer dereference a1 for NUMBER)
\end{itemize}
     
\subsection{Matrices}
The conventions for Matrices are similar to those for
arrays:
\begin{tt} \begin{verbatim}
     PROCEDURE myproc(matrix1,matrix2:Matrix;.......
     (*matrix1 and matrix2 should be mxn Matrices*)
     VAR
       m:=NUMBER(m1^);    m1:=0;  mm:=m-1;
       n:=NUMBER(m1[0]);  n1:=0;  nn:=n-1;
     BEGIN
       IF NUMBER(matrix2^)#m OR NUMBER(matrix2[0])#n THEN
         RAISE Error(Err.bad_size);
       END;
     
       FOR i:=m1 TO mm DO
         FOR j:=n1 TO nn DO
           (*do something to matrix1 and matrix2*)
         END;
       END;
     END;
\end{verbatim} \end{tt}
