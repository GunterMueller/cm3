\section{xCmplx: Complex Numbers}
This module has had an interesting history.  It started as a reimplementation
of NR92's Appendix C.  I [HGG] didn't know about the section 5.4, which
explains the algorithms, so I had to reverse engineering the code itself.
Once I understood the issues, I reimplemented.  I also looked for
and used alternative references for several of the functions.

Later, I added the transcendentals from Krey88, ch 12, pp765-
766. 

In March of 1996, Warren Smith made contact and submitted a wealth of
new code, some dependent on his own Complex module.  I folded that into
the existing xCmplx, assuring both HGG and WDS conventions could be used.
I removed some of my old code and replaced it with WDS implementations.
Some of the old code I left as is.  Notice that the results depend on
structural equivalence in type checking --- doing this in a language
which required name equivalence would have been grim. 

WDS generally documents his work meticulously in extended comments.  
For other modules I have translated those to \TeX\ formatrix.  
However, for this
module there is relatively little documentation provided, so none is
presented here.

\subsection*{add, sub}
Trivial

\subsection*{mul}
See also Hopk88, pp35-36
\begin{verbatim}
     c3.re:=c1.re*c2.re-c1.im*c2.im;
     c3.im:=c1.im*c2.re+c1.re*c2.im
\end{verbatim}     


\subsection*{div}
This is retained as hgg\_div, but WDS's implementation is exported.

The basic formula (CRC91, pg 337) is [x is re and y is im]:
\begin{equation}
  \frac{(x_1x_2+y_1y_2) + i(x_2y_1-x_1y_2)}
       {x_2^2 + y_2^2}
\end{equation}
However, this takes 8 multiplies.  Also, if x2 and y2 are
different in magnitude, there can be truncation errors.  So,
if x2 is the biggest, we divide top and bottom by 1/x2:
\begin{equation}
  \frac{(x_1+y_1(y_2/x_2)) + i(y_1 - x_1(y_2/x_2))}
       {x_2 + y_2*(y_2/x_2)}
\end{equation}
Notice how the ratio $y_2/x_2$ repeats.  Also, we use the
denominator for both the re and im portions.  We now have 6
multiplies, and the opportunity to prevent at least some
truncations.

\subsection*{abs}
See WDS's Magnitude comments.

The basic formula (CRC91, pg 337) is
\begin{equation}
  |z| = \sqrt{(x^2 + y^2)}
\end{equation}
     
The simple cases of $x=0$ or $y=0$ can be resolved right away.

Otherwise, if x and y are different in magnitude, we can
have truncation error.  In 64-bits, we have $EPS=1e-17$, so
$x/y$ (or $y/x$) would have to be that extreme to cause a
problem.  I'll take that chance.

If the problem does arise, or if one needs a 32-bit
representation (where the trouble arises much sooner), do a
normalization.  See also NR92:
\begin{verbatim}
     x:=|x|; y:=|y|;
     if x > y then r=y/x and:
     
     |z| = (x^2+y^2)^1/2
         = [x^2*(1+y^2/x^2)]^1/2
         = x*(1+(y^2/x^2)^1/2
         = x*(1+(y/x)^2)^1/2
         = x*(1+r*r)^1/2, x > 0
\end{verbatim}
          
Now we have $1 + something\_less\_than\_one$, which might still
give a truncation error but has a better chance than just
adding the squares.  It costs an extra multiply.  Of course,
if $y >x$, then the analysis is reversed.
\begin{verbatim}
     if y>x then r=x/y and:
     
       |z| = (x^2+y^2)^1/2
           = [y^2*(x^2/y^2+1)]^1/2
           = y*(x^2/y^2 +1)^1/2
           = y*((x/y)+1)^1/2
           = y*(r^2+1)^1/2, y > 0
\end{verbatim}

\subsection*{sqrt}
See WDS's Sqrt comments.

The basic formula (Krey88, pg 730, eqn 17) is
\begin{equation}
  \sqrt{z} = \pm \left[ \sqrt{\frac{1}{2}(|z| + x)} +
     \mbox{sign}(y)*i*\sqrt{\frac{1}{2}(|z| - x)}\right]
\end{equation}

This seems to do ok.  NR92, pg177 provides a different set
of formulas.  When time permits, we should do accuracy and
timing comparisons.

\subsection*{exp, ln}
From Krey88.

\subsection*{powN, powXY}
From Krey88.


\subsection*{cos, sin, tan}
For cos, the formula is:
\begin{equation}
  \cos(z)=\cos(x)\cosh(y) - i \sin(x)\sinh(y)
\end{equation}

Where:
\begin{equation}
  \cosh(x)=\frac{1}{2}(e^{x} + e^{-x})
\end{equation}

Or:
\begin{verbatim}
     ex:=exp(x); ey:=exp(y);
     tmp.re:=+0.5*cos(x)*(ex-1.0/ex)
     tmp.im:=-0.5*sin(x)*(ey-1.0/ey)
\end{verbatim}

Clearly there is opportunity for truncation in the ex and ey
terms.  Is it a problem? Taking the ex term to be specific:
There is truncation if the exponents of $e^x$ and $e^{-x}$ differ
by more than 17.  This the exponent of $e^x$ can be up to $1/2$
of this or $17/2 = 8$.  Thus x can be up to $\ln(1E8)=~18$.  By
using matrixhcad, we find we can get to $x=~20$ before blowing
up.

Alternatively, factor out $e^{-x}$:
\begin{verbatim}
     tmp.re:=+0.5*cos(x)*(ex*ex+1.0)/ex;
\end{verbatim}

This may or may not be more stable, but doesn't fundamental
shift x's range.  We are still in the $|x|<=18$ range.

Is that enough?  CRC91 provides table values to $x=10$.  Let's
accept 18 as the top end, document this, and report an
exception if we go over that.  This will be true of both the
x and y inputs.
\begin{tt} \begin{verbatim}
     IF ABS(c.re) > 18.0D0 OR ABS(c.im)> 18.0D0 THEN
       RAISE Error{Err.out_of_range};
     END;
\end{verbatim} \end{tt}

sin is of course similar.  Notice that for sin and cos there
are no opportunities for ex and ey reuse.  We may as well
use the Math cosh, sinh functions (hoping they are more
robust than our own analysis here).

tan is $sin/cos$:  In this case we can probably reuse ex, ey
from the various cosh and sinh calculations. For now we
however, we will do it naively.


\subsection*{cosh, sinh, tanh}
The formula is:
\begin{equation}
  \cosh(z)=\frac{1}{2}(e^{z}+e^{-z})=\cos(i z)
\end{equation}
     
Where $e^z$ is found by
\begin{verbatim}
       ex:=exp(z.re);
       tmp.re:= ex*cos(z.im);
       tmp.im:= ex*sin(z.im);
\end{verbatim}

Once again, we can do a full analysis, or just take the easy
way out.  If we ever get into a problem space requiring
hyperbolics in an inner loop, we should do timing analysis.
For now we will take the easy way out:
\begin{verbatim}
     cosh(z) = cos(i*z) = cos(i*z.re + i*i*z.im) = cos(-z.im + i*z.re)
     
     tmp.re:=-z.im;
     tmp.im:=+z.re;
     tmp:=C.cos(tmp);
     return tmp;
\end{verbatim}

sin is similar:
\begin{verbatim}
     sinh(z) = -i*sin(i*z) = -i*(-z.im + i*z.re)
     
     tmp.re:=-z.im;
     tmp.im:=+z.re;
     tmp:=C.sin(tmp);
     (*tmp.re = -i*i*tmp.im = tmp.im*)
     (*tmp.im = -i*tmp.re* = -tmp.re*)
     t:=tmp.im;
     tmp.im:=-tmp.re;
     tmp.re:=t;
     return tmp;
\end{verbatim}

\subsection*{pmul, pdiv}
If we happen to be in polar form, then multiplication and
division are simple.  For pmul:
\begin{verbatim}
     tmp.radius:=p1.radius*p2.radius;
     tmp.angle:=p1.angle+p2.angle
\end{verbatim}

For pdiv:
\begin{verbatim}
     tmp.radius:=p1.radius/p2.radius;
     tmp.angle:=p1.angle-p2.angle
\end{verbatim}

In addition, we need to normalize the angles to the $-\pi \dots +\pi$
range:
\begin{tt} \begin{verbatim}
     WHILE tmp.angle < -Pi DO
       tmp.angle:=tmp.angle + TwoPi;
     END;
     WHILE tmp.angle > Pi DO
       tmp.angle:=tmp.angle - TwoPi;
     END;
\end{verbatim} \end{tt}

NOTE:  {\tt arg} and {\tt toPolar} use {\tt atan2}, which does
the normalization itself, so we don't need to handcraft
normalization code for them.
