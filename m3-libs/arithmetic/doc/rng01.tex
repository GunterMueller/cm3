\section{RNG01: Random Number Generators -- 01}

\subsection*{DECSRC}
This is just a wrapper for the DEC SRC-supplied {\tt Random.T.longreal}.

\subsection*{ran0}
NR92, pg 278, describes Schrage's approximatrixion to Park and Miller's
proposed {\em Minimal Standard} generator. The basic approach is:
\begin{verbatim}
     a) z = a*(z mod q) - r*(z div q)
     b) If z <0 then z = z+m
     c) To get back to 0.0 --> 1.0, result = float(z) / float(m)
\end{verbatim}
     
{\tt mod} is tricky, because you have to do at least:
\begin{equation}
  z \bmod q = z - (z \mbox{div} q)*q
\end{equation}
And maybe deal with negatives.  I suspect M3's MOD function
is fairly complex compared to a simple idiv.  So we ought to
take advantage of knowing z is positive, and do our own mod.

Notice that z div q can be pulled out as a temp.:
\begin{eqnarray}
     tmp & = & z \mbox{div} q\\
     z & = & a*(z - \mbox{tmp}*q) - r*\mbox{tmp}
\end{eqnarray}

Next, remember that dividing by float(m) is more expensive than
multiplying by 1.0/float(m), which can be done once at
initialization:
\begin{verbatim}
     Mrecip = 1.0 / float(m)
     ...
     result = float(z) * Mrecip
\end{verbatim}
     
Not too surprisingly, this is what NR92 uses.  To claim original
code, we can use one of the alternate sets of constants.
We'll use the first alternative.

NR92 makes a point of telling us to not include the
endpoints (0.0 and 1.0).  Thus there is a Min value and a
Max value.  Min should be such that we don't get a 0.0 when
captured in REAL64 format.  Min can be a few times bigger than EPS.
Max is then that close to 1.0, or (1.0-Min), which can be precalculated.

\subsection*{ran1}
From NR92.  The trick here is to use the ran0 stuff, but use that
result to access a shuffle table.  The shuffle table (per
NR92, pg 281, fig 7.1.1) requires us to take the result
modulo the table size.  But as noted above, mod can be done
inline.  Use that to get the shuffled value, and put the old
one in its place.
\begin{verbatim}
     ndx:=z - (z div size)*size
     z2:= table[ndx]
     table[ndx]:=z
\end{verbatim}
     
Bays-Durham further cooks the data by saving z2 and using it
for the ndx computation.  That means we have to save it and
have to initialize it.  NR92 initializes it with the first
table value.  Just to be different, we'll use the 3rd table
value.  NR92 also warms up the generator with 8 throwaways.
We'll pass on that.



