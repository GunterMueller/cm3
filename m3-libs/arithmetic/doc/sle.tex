\section{xSLE: Simultaneous Linear Equations}
There are many, many routines which are useful for systems of
linear equations.  We only catpure some of the common forms here.

\subsection{Tridiagonal Form}
The tridiagonal for can be solved by a forward pass and a backward pass.
It is therefore worth our while to find ways to get matrices into that
form.  

\subsection*{householder}
This implementation is taken from Naka91, pg 263.

\subsection*{matrix\_to\_arrays}
This is a convenience function, in case you have the full matrix form
(e.g., from householder) and need to convert to the reduced form.

\subsection*{tridiag}
This one took a while to work out.  I started with
the obvious conditions of A=1x1, 2x2, 3x3.  But by stopping
there, I couldn't see the pattern.  Only when I handworked
the 4x4 case, did the recurrence formula stand out.  The
clearest case in the 3rd row in a 4x4.

We solve each row to this form, where ut2 is known and d2*u3
is (obviously) unknown.
     u2=ut2 - d2*u3

Then proceed with row 3:
     a3*u2 + b3*u3 + c3*u4 = r3
     
     u3 = (r3-a3*u2-c3*u4)/b3
     u3 = (r3-a3(ut2-d2*u3)-c3*u4)/b3
     u3*(1-a3*d2/b3) = u3*(b3-a3*d3)/b3 = (r3-a3*ut2-
     c3*u4)/b3
Multiply both sides by b3, then divide by the left factor,
which we call den for denominator:
     den = (b3-a3*d3)
     u3 = (r3-a3*ut2-c2*u4)/den
Note that part of this is solvable, and part is not, so
split the pieces:
     u3 = (r3-a3*ut2)/den - c3*u4/den
Resolve as far as possible:
     u3 = ut3 - d3*u4
We are now in the same form that we started from u2, and are
ready to proceed with the next row.  This continues until
the last row, at which time there is no c(n)*u(n+1) term,
and we can solve directly.  At that point, we have a vector
of ut's, but have to subtract out the d(n)*u(n+1) terms.
That means a) we have to save the d terms, and we have to
work backwards from u(n) to u(1).

Actually, we can cheat a bit and use the existing u vector
as the ut vector, since we never need both values at once.
So we load up the u vector going forward, and then subtract
out the d*u terms going backwards.

