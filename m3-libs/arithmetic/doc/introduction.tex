\section{Introduction}

\subsection{Purpose}

\newcommand\nadt[2]{\item \texttt{#1} - #2}

The \modna{} library is a collection of arithmetic operations
on a range of mathematical objects:
\begin{enumerate}
\nadt{Integer}{machine size integers}
\nadt{BigInteger}{integers of arbitrary size}
\nadt{Float}{machine oriented floating point numbers}
\nadt{Complex}{complex numbers}
\nadt{Polar}{2D vectors in polar representation}
\nadt{Fraction}{fractions of arbitrary objects}
\nadt{Vector}{vectors of arbitrary size}
\nadt{Matrix}{}
\nadt{Polynomial}{}
\nadt{Root}{roots of polynomials}
\nadt{PhysicalValue}{numbers equipped with physical units}
\end{enumerate}
It includes algorithms for
special functions,
fourier transform,
interpolation,
root finding,
random numbers
as well as routines for generating \LaTeX{} output and
function plots.

All datatypes presented except of the floating point numbers
allow for exact computation.
That's why one can no longer talk of a library
that is specialized for numerical analysis.
Though any instantiations of one
of the composed datatypes with floating point numbers
requires algorithms specialized for the problems
of numerical operations.

There are many libraries around
that deal with mathematical operations,
computer algebra and function plotting,
so what is the sense of a new library of this kind?
The reason is simply to provide
a mathematical tool in the static type safe, logical, comfortable
but efficient programming environment of Modula 3.
Even expensive computer algebra tools like Maple or Mathematica
or wide-spread numeric tools like MatLab
can't provide static type safety.
Instead they declare the missing static typing as advantage
but in practice it proves to be a source of many programming bugs.
Same applies for scripting languages like Python and Perl.
C++ and Ada may seem to be alternatives because they provide
overloading of operators.
In fact this technique is syntactic sugar
and if heavily used it let the user completely lose control
of which functions are actually called.
The unlogical syntax of C prohibits Java and again C++.

The only alternative for programming languages
suitable for mathematical computations or
programming in general is Haskell and
similar functional languages.
Haskell provides static type checking
with easy accessible polymorphism
while short programming texts
but reducing the user's control over
time and memory consumption.

\subsection{History}

Modula 3 numerical analysis library was started in 1996 and
maintained sinced them by Harry George.
Many people contributed to it
but for many years there was only few or even no further activity on it.
In 2003 I (Henning Thielemann) started on this code
and restructured the modules to be more generic.
E.g. now it is possible to combine each data type with almost each other,
say polynomials with complex numbers as coefficients.
Module templates are used to achieve this flexibility
while being efficient in runtime but unfortunately not in code size.

The restructured \modna{} library preserves all of the previous functions
but it is no longer compatible with the original version.
I hope that the greater flexibility will excuse this.

\subsection{Presence}

Because of the heavy restructuring of the library
this document is currently not conform
to the contents of the modules, unfortunately.
Also some funny words may have been produced by stupid
find/replace of the old module names by the new one.

The library is still evolving and
due to the currently small user base
no one can foresigh if the API is appropriate
in the general case.
That's why everything should be handled with care and scepticism.
Still the API may change in order to get a better structure!

\subsection{Future}

For the documentation I suggest the usage
of a documentation extraction tool
to simplify updates of modules and their documentations.
This tool should capable of typesetting formulas
with the quality of \LaTeX{}.

Further extensions of the library will contain interfaces
to \hyperref{LAPACK}{http://www.netlib.org/}
to make use of well tested state-of-the-art
numeric linear algebra algorithms
as well as to the function plotting package
\hyperref{plplot}{http://plplot.sourceforge.net/}.
This way we can make use of advantages in this libraries
while accessing them through a safe and comfortable interface.
