\section{Introduction}

\subsection{History}

Modula 3 numerical analysis library was started in 1996 and
maintained sinced them by Harry George.
Many people contributed to it
but for many years there was only few or even no further activity on it.
In 2003 I (Henning Thielemann) started on this code
and tried to get a more flexible structure to the modules,
i.e. it is possible to combine each data type with almost each other,
say polynomials with complex numbers as coefficients.

\subsection{Remarks}

This document is currently not conform
to the contents of library, unfortunately,
because the library has been restructured heavily.
Now the question is
if one continues with a separate TeX document
or if one uses one of the documentation systems
that extracts the documentation from the source code.
Some funny words may have been produced by stupid
find/replace of the old module names by the new one.



conventions:
quake templates, proc parameters
order of generic formals
common abbreviations of module names
module names (fraction of polynomials or polynomials of fractions?)
don't rely on order of field names, e.g. in TexStyle, FmtStyle

directory structure
structure of each module, common functions
