\section{xRNG02: Random Number Generators -- 02}
[Warren D. Smith provided the analysis and the generators.]

The goal is to combine several random number generators to get at least the 
randomness of the strongest of them, but to do it in such a way as
to get nearly the speed of the fastest of them.

NOTE: I use a lot of global variables here to save state,
necessary in modula3 since C's "static" feature does not exist.
Of course these variables are local to the current module.
For that reason, please keep this a seperate module from others,
e.g. the derived routines for NormalDeviate, etc. should not be in
same module as these underlying routines for uniform deviates.

Some component generators:
Following G.Marsaglia: A current view of RNGs, pages 3-10 in
Computer Science and Statistics, the interface, Elsevier 1985,
a fair number of my generators are of the FIBO(a,b,\%) type, where
$a>b>0$ are integers such that $x^a+x^b+1$ is a primitive trinomial
mod 2 and \% is a binary operation.

Suitable values of (b,a) are tabulated in D.Knuth: Seminumerical methods,
Addison-Wesley 1981, page 28, which is extracted from larger tables in
N.Zierler \& J.Brillhart: Information and Control 13 (1968) 541-554;
14 (1969) 566-569; 15 (1969) 67-69. Some pairs with "a" prime which
go beyond the Knuth table, which you can use if you *really*
want randomness, include: 
\begin{verbatim}
  (7,127), (15,127), (30,127), (63,127),    (32,521), (158,521), 
  (105,607), (147,607), (273,607),     (216,1279), (418,1279)
\end{verbatim}

The binary operation \% could be: [- mod 1] for floating point values,
[- mod M] or [* mod M] for some integer M,
or [this NOT recommended; it is same as - only mod 2] XOR.

Then the meta-procedure Fibo(a,b,\%) is as follows (it
requires an auxiliary arr : ARRAY [0..a-1], initially filled with
random seed values, and initially ia=a and ja=b:

\begin{verbatim}
MetaRandGen() : Type =
  BEGIN
    DEC(i);
    DEC(j);
    IF i<0 THEN
      i := a-1; (* wraparound *)
    ELSIF j<0 THEN
      j := a-1; (* wraparound *)
    END;
    arr[i] := arr[i] \% arr[j];
    RETURN arr[i];
  END MetaRandGen;
\end{verbatim}

Period [Marsaglia \& Tsay:  Lin. Alg. and its
   Appl. 67, 147-155, 1985]: If \% is $(- \mbox{mod} 2^{\mbox{wordsize}})$,
then if at least one of the arr[] is odd
the period will be maximal $= (2^a - 1) * 2^{(\mbox{wordsize}-1)}$ if
the characteristic axa matrix T of the recurrence has full order
$J=(2^a - 1)$ in mod 2 arithmetic (this assured by the trinomial condition)
and 2*J in mod 4 and 4*J in mod 8 arithmetic. That is, if you square
T a times, you get back T in mod 2, but do NOT get back T in mod 4,
arithmetic, and if you square T a+1 times, you do not get back T
in mod 8 arithmetic. In fact, this criterion will work for any
matrix T, not necessarily the fibo($-$) one...

If \% is $[* \mbox{mod} 2^{\mbox{wordsize}}]$, which may be implemented in m3 by Word.Times(),
then arr[] must be all odd integers, and by considering discrete logs
we see that the period is maximal $= (2^a - 1) * 2^{(\mbox{wordsize}-3)}$ if
the $F(a,b,+ \mbox{mod} 2^{\mbox{wordsize}})$ has maximal period
$(2^a - 1) * 2^{(\mbox{wordsize}-1)}$.

Marsaglia's test results for Fibo(a,b,\%) generators with
(a,b)=(17,5), (31,13) and (55,24):
\begin{description}
\item[XOR] With \%=XOR the fibo fail seven of the tests in Marsaglia's battery.
As Marsaglia says in his conclusion "never use XOR." Despite this,
these generators keep resurfacing, for example the "r250" Fibo(250,147,XOR)
generator of S. Kirkpatrick and E.  Stoll, Journal of Computational
Physics, 40, p. 517 (1981) and W. L. Maier, "A Fast Pseudo Random
Number Generator", Dr. Dobb's Journal \#176.

\item[Subtraction] With \%=-, among the tests in Marsaglia's DIEHARD battery
only the "birthday spacings test" (of frequencies of spacings in sorted sets 
of deviates) is failed, a symptom both of the linear structure
and also of the specific subtractive structure of these generators.
Note that the Fibo(55,24,-) generator (which fails this test) is to be found
in Knuth's book and also is distributed with the DEC SRC implementation of
modula-3. Also note that postprocessing Fibo(55,24,-) sequence with a
Bays-Durham shuffler, recommended in Press et al. "Numerical Recipes" to
fix suspicious generators, will NOT work to make it now pass this test, since
the birthday spacings test does not depend upon the ordering of the
deviates only their values.

\item[Multiplication] With \%$=* \bmod 2^{32}$, these
FIBO gens passed all the tests in
Marsaglia's battery.
\end{description}

Note that the -,+ and XOR gens are "linear" and hence theoretically
bad and will always fail "empty slab tests" see below.

Note that shift register, Fibonacci() generators with +,- or XOR,
and linear congruential generators x <-- a*x+b mod M, AND linear
combinations of such possibly to different moduli, are all
"linear" and thus generate d-tuples of random numbers lying on
AT MOST $M^{(1/d)}$ hyperplanes in d-space. Consequently,
the nonrandomness of a linear RNG with period M is in principle
detectable by the "largest empty slab" test in .4*logM dimensions
after only N,
\begin{verbatim}
  N > e^(5/2) * (.4*logM)^2.5 * loglogM,
\end{verbatim}
random numbers have been generated. Hence:
Please do not rely on a linear generator.
If you are going, foolishly, to use a linear congruential generator, though,
you want a "good multiplier" a mod M. (Bad multipliers
will result in even fewer hyperplanes, for example IBM's
infamous RANDU generator generated points lying on only 15 planes in 3-space.
Even with fairly good multipliers we still get test failures, e.g. Marsaglia's
spectrally good multiplier $69069 \bmod 2^{32}$ fails his OPSO test, as does
the Berkeley PASCAL $62605 \bmod 2^{29})$. It will suffice, for
comparatively good behavior in d-space, that
\begin{verbatim}
   (a^d mod M)/M
\end{verbatim}
have only small partial quotients in its continued fraction expansion.
This should be tested for $d=1,2,..,8$ at least.

Thus for example, consider the prime modulus $M=2^{35} - 849$.
$(M-1)/2$ is also prime. I found the generator $g=145683$ by computer
search. The continued fraction expansions
\begin{verbatim}
CF(g/p)=[235852,1,3,6,1,1,5,1,1,2,1,1,1,1,6,2]
CF(g^2/p)=[1,1,1,1,1,1,1,1,21,2,7,1,8,1,2,1,1,2,5,1,1,1,2,1,2,3,12,2]
\end{verbatim}
are rather nice, and the $CF( (g^x \bmod p)/p )$ for $x=3,4,5,6,7,8,9$ are
also not bad (no partial quotient larger than 23):
\begin{verbatim}
 [2,1,1,1,1,1,1,1,1,16,1,3,5,1,2,1,1,3,1,1,1,21,1,8,10,5]
 [8,4,3,1,1,3,4,2,7,1,22,22,9,1,12,1,2,2]
 [1,1,1,20,1,23,1,4,2,2,1,1,1,8,1,1,2,1,2,2,1,22,1,2,1,3]
 [1,1,15,1,1,10,2,2,2,1,4,5,1,2,1,1,5,7,7,2,2,4,2]
 [1,1,5,2,2,1,1,1,7,11,3,4,1,9,1,6,3,1,8,1,3,1,1,1,2]
 [1,13,1,7,8,1,1,2,16,2,2,1,1,6,4,1,3,1,1,3,15,2]
 [2,1,2,1,15,2,1,5,1,8,1,1,2,1,4,3,1,3,1,1,1,1,2,1,3,1,3,5]
\end{verbatim}
so we conclude that using g as a multiplier should exhibit comparatively
good behavior in dimensions 2-9. This particular (g,M) pair
has the virtue that IEEE doubles can represent integers up to and including
$2^{53} - 1 $exactly, so that the modula-3 statement
\begin{verbatim} x := (g*x) MOD M; \end{verbatim}
will evaluate it exactly.

As another example, the Marsaglia multiplier
$69069 \bmod 2^{32}$, while spectrally good in
dimensions 2-5, is bad in dimension 6, as is revealed by the spectral
test directly but also simply by noticing that the CF expansion of
$69069^6 / 2^{32}$ is [1, 75, 1, 2, 2, 1, 20, 10, 3, 10, 2, 2, 1, 12, 9]
which contains the large number 75 early on.

You also probably want
full period M, which happens if [thm page 16 Knuth]
\begin{verbatim}
  (1) GCD(b,M)=1;
  (2) a-1 is a multiple of p for every prime p dividing M;
  (3) a-1 is a multiple of 4 if 4 divides M.
\end{verbatim}
If M is prime and b=0, you get maximal period M-1 if a primitive mod M.
If M>=16 is power of 2 and b=0, get maximal period M/4 if a=3 or 5 mod 8.

\subsection*{Other Fibo generators}
Instead of just using one lag term and binary operation \%, you could combine
with TWO lag terms via some TERNARY operation, or THREE lag terms
via a QUATERNARY operation, etc. I suggest the new generator
\begin{verbatim}
  QuaternaryFibo(a,b,c,d, x0 - (x2 XOR (x1 - x3) mod M) mod M ).
\end{verbatim}

If M is $2^{\mbox{wordsize}}$ and $x^a+x^b+x^c+x^d+1$, $a>b>c>d>0$,
 is a primitive
polynomial mod 2, then this generator's period will be at least $2^a - 1$
simply by considering the LS bit, which will follow a DeBruijn sequence
and thus exhibits good randomness in $\le a$ dimensions.
My idea is that by using
both XOR and +, we hope to avoid the weak behavior of either operation
alone, e.g. with respect to the birthday spacings test. This may also
make the generator "nonlinear" (i.e. not outputting a lattice), a point
I am unsure about. Ternary generators of this type do not seem to exist since
4-term primitive polynomials $x^a+x^b+x^c+1 \bmod 2$ do not exist (you
need an odd number of terms, as can easily be shown).
For tables of examples of primitive polynomials mod 2, see E.J.Watson: Math. of
Comput. 16 (1962) 368-9 but the ones given are the lexically first examples,
bad for our purposes.

Hence I constructed my own small table below of examples of primitive
polynomials $P=x^a+x^b+x^c+x^d+1 \bmod 2$, where a is prime and
$N=2^a - 1$
is a Mersenne prime. For such a, P is primitive if $P^N$
divides $x^N-1 \bmod 2$,
which you can test by the algorithm
\begin{verbatim}
  Q := x;
  FOR i:=1 TO a-1 do
    Q := Q*Q*x mod P;
  END;
  IF Q=1 THEN "P is primitive."
Table:
x^19 + x^9 + x^4 + x^3 + 1
x^19 + x^6 + x^2 + x^1 + 1
x^31 + x^22 + x^20 + x^8 + 1
x^31 + x^16 + x^11 + x^6 + 1 
x^61 + x^5 + x^2 + x^1 + 1 
x^61 + x^57 + x^2 + x^1 + 1 
x^61 + x^19 + x^16 + x^9 + 1
x^89 + x^36 + x^2 + x^1 + 1
x^89 + x^57 + x^14 + x^5 + 1
x^89 + x^28 + x^26 + x^9 + 1
x^89 + x^37 + x^5 + x^1 + 1
x^107 + x^53 + x^45 + x^12 + 1
x^107 + x^93 + x^68 + x^2 + 1
x^127 + x^29 + x^17 + x^7 + 
x^127 + x^67 + x^65 + x^17 + 1
x^127 + x^60 + x^22 + x^9 + 1
x^521 + x^46 + x^38 + x^37 + 1
x^521 + x^249 + x^92 + x^87 + 1
x^521 + x^353 + x^258 + x^6 + 1 
x^607 + x^66 + x^60 + x^30 + 1 
x^607 + x^247 + x^83 + x^71 + 1
x^607 + x^483 + x^449 + x^298 + 1 
x^607 + x^442 + x^113 + x^23 + 1 
\end{verbatim}

It is a disgusting fact that in modula3 as well as most high level languages,
the machine language multiply instruction that computes a*b to double
precision is NOT ACCESSIBLE. Ditto "add with carry". This makes an
efficient implementation of high precision multiplication (\& modular
version) obnoxiously difficult and inefficient. And you really
do need multiple precision to get a decent period with a lincong
or iterated squaring generator.
[Note the SQUARE ROOT of the period must be unreachably large for
good randomness, NOT the period, a common error.]
Here is a MetaAlgorithm which works
if numbers 0..modulus*2-2 are representable...:
\begin{verbatim}
  MetaModularMultiply(x,y, modulus: NumType) : NumType =
  VAR
    q : NumType := x;
  BEGIN
    FOR b = bits of y in MS-->LS order starting
        at the bit AFTER the first 1 bit DO
      q := q+q;
      IF q>=modulus THEN q := q-modulus; END;
      IF b=1 THEN
        q := q+x;
        IF q>=modulus THEN q := q-modulus; END;
      END;
    END;
    RETURN q;
  END MetaModularMultiply;
\end{verbatim}

One way to improve the randomness of a generator is
C.Bays and S.D. Durham's shuffling algorithm, ACMTOMS 2 (1976)
59-64, also described in D.Knuth: Seminumerical algorithms.
Knuth says this will output a "considerably more random" sequence.
However, the set of values will still be the same (although ordered
differently) hence, e.g., the subtractive and XOR fibo gens will
STILL fail Marsaglia's birthday spacings test even after improvement
by shuffling. However, linear congruential generators are probably
substantially improved by shuffling since the lattice structure is
destroyed.

L.Blum, M.Blum, M. Shub: A simple unpredictable psuedo-random
number generator, SIAM J. Comput 15,2 (1986) 364-
following
A.Yao: Theory and applications of trapdoor fns, 23rd SFOCS (1982) 80-91
and
W.Alexi, B.Chor, O.Goldreich, C.P.Schnorr: 25th SFOCS (1984) 449-457.
point out that iterated squaring generator
\begin{verbatim}
  x <--- x*x mod M
\end{verbatim}
where M=p*q, p, q primes that are 3 mod 4, will have these properties
(assuming it is computationally infeasible to compute the factorization of M)
\begin{enumerate}
\item infeasible to predict the PREVIOUS x.
\item infeasible to predict the least signif bit of the
    previous x with correctness prob $>$ 1/2 + 1/polynomial.
\item infeasible to predict the boolean "x<(M-1)/2" for the
    previous x with correctness prob $>$ 1/2 + 1/polynomial.
\end{enumerate}
Consequence of (2) is: no polynomial time statistical test
can invalidate the randomness of the sequence of LS bits of
the iterated squaring generator. 

The full iterated squaring gen, not just its LS bits,
is a lot faster and (conjecturally!) just as random.
Its period is (p-1)*(q-1)/4.

Example: Here is a product of two primes both 3 mod 4:
\begin{verbatim}
 94906247 * 94906219 = 9007193062250093 = 2^53 - 6192490899.
\end{verbatim}
Here is a smaller example:
\begin{verbatim}
    M = 9739 * 9719 = 94653341.
\end{verbatim}
This modulus has the virtue that iterated squaring mod M
may be accomplished exactly in IEEE double arithmetic using
\begin{verbatim}
 x := x*x MOD M, since M^4 < 2^53 - 1.
\end{verbatim}
An even smaller modulus suitable for 32-bit unsigned arithmetic is
\begin{verbatim}
    M = 239*251; (* = 59989; 239 and 251 are each primes and 3 mod 4 *).
\end{verbatim}

More generally you could use iterated cubing (subject to the conjecture
it is infeasible to break any bit of the RSA cryptosystem) or in fact
any fixed exponent you want (under same conjecture) and in fact almost
any fixed integer linear operation on the discrete logarithms of a vector
will by the same reasoning be immune to any polytime statistical test,
subject to RSA-type conjectures. In particular, we conjecture that the
Fibo[a,b,*] generator ought to be immune to polytime statistical tests
if the modulus is the product of two suitable large primes... which
explains the results of the Marsaglia test battery above.





