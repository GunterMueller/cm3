\section{xPoly: Polynomials}
We will use the $[a_0,a_1,a_2 \dots]$ format for the array
representation of polynomials.  That allows the array index
to be the x exponent.  However, it makes the bookkeeping for
div and deriv a bit tricky, since they have to work
backwards.

\subsection*{eval}
NR92 recommends using the nested form:
\begin{equation}
    a_0 + a_1 x^1 + a_2 x^2 + a_3 x^3
      = a_0+ x(a_1+x(a_2+x(a_3)))
\end{equation}

This can be done in a loop, working down from a[nn]:
\begin{tt} \begin{verbatim}
     tmp:=p[nn];
     FOR i:=nn-1 TO 1 BY -1 DO
       tmp:=a[i]+x*tmp;
     END;
     tmp:=a[0]+tmp;
\end{verbatim}  \end{tt}
   
\subsection*{add,sub}
Trivial.

\subsection*{mul}
This is not as simple as add and sub, but it it is still
mainly bookkeeping.  No special insights needed.

\subsection*{div}
The classic form is:
\begin{equation}
   \frac{A=a_n x^n+a_{n-1} x^{n-1}+\dots+a_0}
     {B=b_m x^m+b_{m-1} x^{m-1}+\dots+b_0}
\end{equation}

We set it up for longhand as:
\begin{tt} \begin{verbatim}
          Q=quotient
          ------------------
     B    | A
           -B*q[qi]
            --------------
            remainder
\end{verbatim} \end{tt}

We repeat a pattern, using $b[m]=bmax$ as the divider, getting
q's in decreasing order and taking the remainder of the
subtraction as the start point for the next division by
bmax.  We may as well start by copying A into the remainder
and using it directly:
\begin{verbatim}
     q[qi]:=r[ri]/bmax
     for all the r's, subtract the appropriate q[qi]*b[bi]
\end{verbatim}

The problem is really one of bookkeeping, since we are
moving backwards in b,r,and q, and the start points keep
changing.  Points to remember include:
\begin{enumerate}
     \item Once the order (i.e., max exponent) of the remainder
       is less than bmax, we can quit.
     \item Q's order is A's order - B's order.
\end{enumerate}
     
Using $A=u, B=v$:
\begin{tt} \begin{verbatim}
     (*u/v = q, remainder r*)
     un:=NUMBER(u^); unn:=LAST(u^);
     vn:=NUMBER(v^); vnn:=LAST(v^); vmax:=v[vnn];
     qn:=un-vn+1; q:=NEW(Poly,qn);
     r^:=u^;
     
     FOR ri:=unn TO (unn-qnn) BY -1 DO
         qtmp:=r[ri]/vmax;
         q[ri-vnn]:=qtmp;
         ri2:=ri;
         FOR vi:=vnn TO v0 BY -1 DO
           r[ri2]:=r[ri2]-qtmp*v[vi];
           DEC(ri2);
         END;
       END;
\end{verbatim} \end{tt}

That is, we work systematically down through the remainders.
For each we use vmax to find q[qi], where qi started at un-
vn and decreases with the r's (which mirror the u's).  Once
q is found, we use it (in the form qtmp) to calc the
subtraction term, which starts with the biggest v and works
down to the smallest.  For each v[vi], the applicable r term
starts with the currently biggest ri (given by ri2), then
decreasing ri2 for every vi decrease.

Testing via mathcad
\begin{equation}
   \frac{1+2x+3x^2+4x^3+5x^4}{.1+.2x+.3x^2+.4x^3} 
     \rightarrow q=.6+12.5x, r=.9+.6x+.3x^2
\end{equation}
    
Multiplying back and adding the remainder I got
\begin{equation}
     u= .96+1.97x+2.98x^2+3.99x^3+5.0x^4
\end{equation}
Looks like the algorithm works, though the lower coefs are
pretty bad.

\subsection*{deflate}
See Swok90, pg 124.  Given polynomial a, and wanting $a/(x-c)
= b$, where a and b are in the form $a_0+a_1*x+a_2*x^2$, we set up
the longhand form:
\begin{verbatim}     
            a[n]n*x^(n-1)
          ------------------------------
     x-c  | a[n]*x^n +  a[n-1]*x^(n-1) ... a0
            a[n]*x^n + -c*a[n]*x^(n-1)
            --------------------
                     (a[n-1]+c*a[n])
\end{verbatim}
     
Notice the $c*a[n]$ term for the next iteration.  Each step
has this addition of c times the previous solution coef.
The result is a very regular progression backwards from $x^n$
to $x^0$.  At that point we cannot divide any further, and just
have the remainder (if any).
\begin{tt} \begin{verbatim}
     m:=LAST(a);
     b:=NEW(Poly,m);  (*b is 0..m-1*)
     b[m-1]:=a[m];
     FOR i:=m-2 TO 0 BY -1 DO
       b[i]:=a[i+1]+c*b[i+1];
     END;
     rem:=a[0]+c*b[0];
\end{verbatim} \end{tt}
     
However, deflation is often used where it doesn't make sense
to generate a brand new "b" polynomial every time.  So we
need an in-place approach.  Since a[] is being used for both
the old input data and the new output data, we have to save
the coef for current term:
\begin{tt} \begin{verbatim}
     (*ann = LAST(a) *)
     b:=a[ann]; asave:=a[ann-1]; a[ann-1]:=b;
     FOR j:=ann-2 TO 1 BY -1 DO
       b:=asave+c*b;
       asave:=a[j]; a[j]:=b;
     END;
     rem:=a[0]+c*a[1];
\end{verbatim} \end{tt}

Deflation is typically used where $c$ is a root of $a$, and
therefore $rem=0$.  So the last line may not be relevant.

\subsection*{deriv}
Given a polynomial with coefs a[]:
\begin{verbatim}
     p =   a0    +a1*x       +a2*x^2     +a3*x^3     +a4*x^4
     p'=   a1    +2*a2*x     +3*a3*x^2   +4*a4x^3    +0
     p"=   2*a2  +2*3*a3*x   +3*4*a4*x^2 +0          +0
\end{verbatim}

By lining them up this way, the polynomial sense of the
vector is maintained.  That is, p[2] is the coef of $x^2$ for p
and p'[2] is the coef of $x^2$ for p'.

Let's start with evaluating p(x).  The NR92 recommended way
to evaluate polynomials is to nest the multiplies.  For p(x)
we have:
\begin{equation}
     p=a_0+x*(a_1+x*(a_2+x*(a_3+x*(a_4))));
\end{equation}
In a loop, this works out as:
\begin{verbatim}
     p:=a[pnn];
     FOR i:=ann-1 TO 0 BY -1 DO
       p:=a[i]+x*p;
     END;
\end{verbatim}

Next, we need to handle derivatives of p.  We may as well
use NR92's notation of pd[], where the index is the
derivative count:
\begin{verbatim}
     p=pd[0]; p'=pd[1]; p'=pd[2]; ...pd[nd];
\end{verbatim}

In nested form:
\begin{verbatim}
     p :=a0+     x*(a1+     x*(a2+     x*(a3+     x*(a4))));
     p':=a1+     x*(2*a2+   x*(3*a3+   x*(4*a4+   x*(0))))
     p":=2*a2+   x*(2*3*a3+ x*(3*4*a4+ x*(0+      x*(0))))
\end{verbatim}

Notice the factors multiplying the a terms.  For the dth
derivative we need:
\begin{equation}
     (n)(n-1)(n-2) \cdots (n-d+1) a_n \mbox{\ (d$<$n)}
\end{equation}

We could do all the derivatives just like p, keeping track
of the offsets and inserting the factors:
\begin{verbatim}
     pd[0]:=a[pnn];
     FOR i:=1 TO nd DO pd[i]:=0.0; END;
     FOR i:=ann-1 TO 0 BY -1 DO
         pd[0]:=a[i]+x*p[0];
       FOR j:=1 TO nd DO
         factor:=1.0;
         FOR k:=i TO i-j BY -1 DO
           factor:=factor*FLOAT(k,REAL32);END;
         pd[j]:=factor*a[i-j]+x*pd[j];
       END;
     END;
\end{verbatim}

But notice something in the nested form.  Starting at the
left, a4 keeps moving to the right as another derivative is
taken.  In fact, the full (factor,a,x ) term is very similar
to the previous line's term one column to the left.  Can we
exploit that to reduce the inner loops in the above
algorithm?

We need to catch the one-lower-derivative before it gets
updated in the current cycle (i.e., the current column).  If
we worked backwards from the largest derivative, we would be
getting both the right a[] term and the right power of x.
Ignoring the factors:
\begin{verbatim}
     p[0]:=a[ann];
     FOR i:=ann-1 TO 0 BY -1 DO
       FOR j:=nd TO 1 BY -1 DO
         pd[j]:=pd[j-1]+x*pd[j];
       END;
       pd[0]:=a[i]+x*pd[0];
     END;
\end{verbatim}

Now, what's happening to the factors?
\begin{verbatim}
   init:
     p"=0; p':=0; p=a4
   iter3:
     p"=p'+x*p"=0+x*0=0
     p'=p +x*p'=a4+x*0=a4
     p =a3+x*p =a3+x*a4
   iter2:
     p"=p'+x*p"=a4+x*(0)=a4
     p'=p +x*p'=a3+x*a4+x*(a4)=a3+2*a4*x
     p =a2+x*p =a2+x*(a3+x*a4)=a2+a3*x+a4*x^2
   iter1:
     p"=p'+x*p"=a3+2*a4*x+x*(a4)=a3+3*a4*x
     p'=p
     +x*p'=a2+a3*x+a4*x^2+x*(a3+2*a4*x)=a2+2*a3*x+3*a4*x^2
     p =a1+x*p =a1+x*(a2+a3*x+a4*x^2)=a1+a2*x+a3*x^2+a4x^3
   iter0:
     p"=p'+x*p"=a2+2*a3*x+3*a4*x^2+x*(a3+3*a4*x)=a2+3*a3*x+4
     *a4*x^2
     p'=p +x*p'=a1+a2*x+a3*x^2+a4x^3+x*(a2+2*a3*x+3*a4*x^2)
          =a1+2*a2*x+3*a3*x^2+4*a4*x^3
     p =a0+x*p =a0+x*(a1+a2*x+a3*x^2+a4x^3)
          =a0+a1*x+a2*x^2+a3*x^3+a4*x^4
\end{verbatim}

Note that p comes out ok (as expected).  Note that p' comes
out ok too (definitely not expected).  p" (and higher
derivatives) are in the right shape, but the factorial terms
are incomplete.  What is the pattern?  Let's set up the
program and try it out.  Use $a[i]=1$ for all i, so we can see
the factorial terms more easily.  Experiments on
$x=1,2,3,4,5$ (comparing to mathcad) show a pattern:  pd[0]
and pd[1] are ok.  pd[1] is off by $2=2!$ and pd[3] is off by
$6=3!$.  In other words all of them are off by a factor:
\begin{verbatim}
     pd[j]:=factorial(j)*pd[j]
\end{verbatim}

It just happens that $0!=1$ and $1!=1$, so pd[0] and pd[1] look
ok.  It appears we can clean this up by putting a factorial
onto pd after computation.  NR92 does this with a home-made
factorial.  Bu this has a hidden integer$\rightarrow$float conversion.
We may as well use the factorial function, which does a
table lookup (at the cost of a funcall).  And let's do it
for all the pd's since it is low cost to catch j=0,1, and
certainly clarifies the analysis.


