\section{xUtils: Utilities}

\subsection*{verbosity, debug}
{\tt verbosity} controls the {\tt debug} function.
0 means don't print anything and 3 means print everything.
It is up to the user of {\tt debug} to
set {\tt level} for each call of {\tt debug}.  E.g.:

\begin {tt} \begin{verbatim}
  (*verbosity has been set to 2*)

  debug(1,ftn,"this will print");
  debug(2,ftn,"this will print");
  debug(3,ftn,"this will not print");
\end{verbatim} \end{tt}

[HGG: I use {\tt debug} because I don't have m3gdb running.]

\subsection*{Error, Err, and err}
The theory is that we carefully define the contract of the interface and
expect the caller to live up to it.  However, in practice, where
it is convenient and not a tremendous overhead we do some defensive
editing.  We also report exceptions that the caller could not
realistically have been expected to know.

We raise just {\tt Error} in this library.
It gets an {\tt Err} value, which can be
used to do case decomposition in the exception handler.

The selection of {\tt Err} values has been haphazard.  We probably should
start with the POSIX math error enumeration, and grow as needed.
Just haven't gotten around to looking at it.

{\tt err} is a shorthand for raising {\tt Error}.
Originally, it also printed
a message to stderr. Since that is no longer the approach, err is not
much used.

 
