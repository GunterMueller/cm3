\section{xRand: Random Numbers}
\subsection{Object Structure}
For most topics we have used simple Modules.  
That is because they did not have state memory.  However, for
the random generators we need to remember data between invocations.  The
selection of which engine, the seed, the current engine settings, etc.
are needed.  Therefore, we wrap the generators in objects.

Many different generators can be used.  See the Random Number Generator
(RNGnn) modules.  Each of these is subtyped from xRand.RandomGen, which
in turn allows the generators to be used to drive various
deviate generators.

\subsection{Deviate Generators}

\subsection*{uniform}
This just returns the engine's value, possibly with scaling to fit
min..max.  It checks to see if the parameters really are different
from Min and Max before bothering to do scaling.

\subsection*{exponential}
This one is trivial.  I don't know what NR92 had in
mind with the do-while loop.  I just return the log, since
I'm already checking for 0.0 in the generators.

\subsection*{gaussian}
We have implemented this twice, with Warren Smith's version being
provided as the default. See the code for his commentary.  

The alternative implementation was based on NR92, pg 289.
The basic equation is:
\begin{equation}
  p(y)dy = \frac{1}{\sqrt{2\pi}}e^{-y^2/2}dy
\end{equation}

The trick is to use the Box-Muller method to transformed y's:
\begin{eqnarray}
     R^2 & = & v_1^2 + v_2^2\\
     y_1 & = & \sqrt{-2.0\ln(x_1)} v_1/\sqrt{R^2}\\
     y_2 & = & \sqrt{-2.0*ln(x_1)} v_2/\sqrt{R^2}
\end{eqnarray}

Since $R^2$ is made from squares, and is supposed to range 0..1, we don't
need to actually do the sqrt.  We just check to see if $R^2$ is
in the 0..1 range.
     
Obviously the sqrt and Rsq can be pulled out for a tmp
value:
\begin{verbatim}
     tmp:=sqrt(-2.0*ln(x1))/R^2
\end{verbatim}

Also, $R^2$ can be used for x1, so let's make a tmp for it:
\begin{verbatim}
     tmp:=sqrt(-2.0*ln(Rsq))/Rsq
\end{verbatim}

Now we get v's until we are inside the unit circle, then use
the radius squared (Rsq) to that point as x1.  We know x1
will be < 1 because we were already inside the circle.  Then
use the tmp factor to get the associated y's:
\begin{verbatim}
     y1:=v1*tmp
     y2:=v2*tmp
\end{verbatim}

The function is only supposed to return one of these at a
time, so we save one for use later, and set a boolean to the
effect that it is ready to use.  Since we already have
"start", we'll use that.  When self.start is true, we need to
generate two more y's.  When it is false, we use the saved
one and set for start again.

\subsection*{gamma}
See W. Smith's code for commentary.

\subsection*{dirichlet}
See W. Smith's code for commentary.

\subsection*{poisson}
Not implemented.

\subsection*{binomial}
Not implemented.

